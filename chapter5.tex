\chapter{Autonomous navigation of micro-magnetic rollers} \footnotetext{This chapter is adapted from unpublished work}
\section{Introduction}
Living cells or bacteria show fantastic ability of navigation called chemotaxis.\cite{alon1999robustness,adler1975chemotaxis} Chemotactic microorganism can trace the very limited noise information in complex micro environment for food source or suitable living space.\cite{keller1971model} As a comparison, although lots of synthetic  micro swimmer can do autonomous directed motion \cite{yan2016reconfiguring,dou2016directed,lee2019directed,baker2019shape}, their motions' direction is biased by  external field or inherent asymmetry instead of information from environment.So the navigation behavior in living system are also highly desirable in synthetic micro swimmers or active matter. \cite{patteson2016active} Only with efficient and accurate navigation behaviors in complex environment, synthetic microswimmers can extend their engineering application  such as drug delivery and surgery.\cite{de2017micromotor,xu2018sperm}

The word "-taxis" refers to the motion following a physical or chemical properties gradient. Different experimental or theoretical research have be reported to make "taxis" synthetic microswimmers. Phototaxis particles can see the gradient of light due to the local thermal gradient near the particle\cite{yu2019phototaxis,dai2016programmable,lozano2016phototaxis,chen2017light}; gravitaxi swimmers can swim against the gravity \cite{campbell2013gravitaxis,ten2014gravitaxis};rheotaxis swimmers can orient their direction autonomously against the fluid flow\cite{Palacci2015,ren2017rheotaxis,brosseau2019relating}; bio-mimic chemotaxis swimmers can sense the local gradient of species and move towards source or drain of chemicals; visoctaxis particles can swim towards the fluid regions of higher or lower viscosity\cite{liebchen2018viscotaxis}. However, these approaches depend on the precise fabrication of microswimmers. Particle of particular shapes or components are designed to show "taxis" behaviors. Different navigation directions can only be achieved by redesigning the particles ,which is a complex and non-efficient process. Less research has been report to navigate a simple and isotropic particles's taxis behaviors. A more programmable and tangible mechanism is needed to navigate large amount of particles autonomously. \cite{dou2019autonomous}

Here, we propose a new gravitaxis mechanism that uses programmed times various magnetic field on isotropic magnetic micro spheres on a inclined surface. Micro spheres in fluid rolling on a surface can couple the rotational motion to transnational motion via hydrodynamics.\cite{galvin2001time,rashidi2016theoretical}. Magnetic micro spheres can response to the external  magnetic showing different rolling dynamics. \cite{helgesen2019propulsion,helgesen2018magnetic}
The external magnetic field provide a rich programmed design space to control the dynamics of magnetic rollers.
In this chapter, we program the external magnetic field exerted on  micro rollers to achieve  with autonomous gravitaxis behaviors on an inclined slope. First we describe a physical model to simulate the motions of magnetic rollers on included surface subject to external magnetic field. Then we explore the design parameter space for magnetic field  and optimize the field leading to desired motions of  micro roller. At the end of this paper, we show some preliminary experiment results as well as the challenge discussion for future experiment. 

\section{Model of  magnetic microrollers}
\begin{figure}[p]
\centering
\includegraphics[width=9cm]{figures/5_1.png}
\caption{ (a) Schematic illustration of a magnetic micro particles with radius$a$, magnetic moment$m$ rolling on an $\alpha$ angle inclined surface subject to a three dimensional magnetic field$B$. Plasma colop is used to represent the height of surface. (b)A simulated trajectory(from left to right) with gravitaxis behaviors on a inclined surface with angle$10$ degree. (c) The applied magnetic field that lead to the trajectory in figure(b). (d) Plot of particle's position in x and y direction versus time.}
\label{fig:1}
\end{figure}
We consider a magnetic sphere with radius $a$ and permanent magnetic moment $\ve{m}$ moving through a viscous fluid at a fixed height above a solid plane under the influence of a time-varying magnetic field $\ve{B}(t)$.(as shown in \textcolor{red}{figure 5.1 a and 5.2a}) In a uniform field, the particle experiences a magnetic torque, $\ve{L}_{\m}=\ve{m}\times\ve{B}(t)$, but no magnetic force, $\ve{F}_{\m}=0$. In the absence of inertial effects (i.e., at low Reynolds number), the magnetic force and torque on the particle are balanced by the hydrodynamic force and torque, which are linearly related to the particle's linear velocity $\ve{U}$ and angular velocity $\ve{\Omega}$ noted as
\begin{equation}
    \begin{bmatrix} \ve{F}_{\m} \\ \ve{L}_{\m} \end{bmatrix} = \begin{bmatrix} \ve{A} & \ve{\tilde{B}} \\ \ve{B} & \ve{C} \end{bmatrix}  \cdot \begin{bmatrix} \ve{U} \\ \ve{\Omega} \end{bmatrix} \label{eq:dynamics}
\end{equation}
where $\ve{A}$, $\ve{B}$, $\ve{\tilde{B}}$, and $\ve{C}$ are components of the hydrodynamic resistance tensor.  For a solid sphere above a solid plane normal to the $z$-direction, the components of the resistance tensor have the form


\begin{equation}
{\scriptstyle
    \ve{A} &=  6\pi \eta a \begin{bmatrix} 
        Y_A & \cdot & \cdot \\
        \cdot & Y_A & \cdot \\
        \cdot & \cdot & X_A  \end{bmatrix}, \quad 
    \ve{B}=-\ve{\tilde{B}}&= 6\pi \eta a^2 \begin{bmatrix} 
        \cdot & Y_B & \cdot \\
        -Y_B & \cdot & \cdot \\
        \cdot & \cdot & \cdot  \end{bmatrix}, \quad 
    \ve{C} =- 6\pi \eta a^3 \begin{bmatrix} 
        Y_C & \cdot & \cdot \\
        \cdot & Y_C & \cdot \\
        \cdot & \cdot & X_C  \end{bmatrix} 
    }
\end{equation}
where $\eta$ is the fluid viscosity. The coefficients $Y_A$ and $Y_B$ describe, respectively, the dimensionless force and torque on a sphere translating parallel to a solid planar surface.\autocite{ONeill1964a} The coefficient $Y_C$ describes the torque on a sphere rotating about an axis parallel to the surface.\autocite{Dean1963} The coefficient $X_A$ describes the force on a sphere translating perpendicular to the surface \autocite{Brenner1961a}.  Finally, $X_C$ describes the torque on a sphere rotating about an axis perpendicular to a the surface. \autocite{Jeffrey1915} These coefficients depend only on the surface separation $\delta$ scaled by the particle radius $a$ as illustrated in Figure \textcolor{red}{5.2 a}.

Substituting the above expressions for the resistance tensor, the linear velocity $\ve{U}$ and angular velocity $\ve{\Omega}$ can be expressed explicitly in terms of the magnetic torque. The dynamics imply that the particle velocity normal to the planar substrate is identically zero (i.e., $U_z=0$). We therefore assume that the surface separation $\delta$ and the resistance coefficients are constant throughout the particle's dynamics. For a spheres on a inclined surface with angle$\alpha$, we can change the coordinate from a world frame to the particle frame as shown in \textcolor{red}{figure 5.2.a}, regarding the rollers is on the a horizontal surface. To simplify the we assume the gravity's influence is negligible with $mg\times a<< m\time B(t)$. The surface face $F_s$ is modeled to balance any force normal to the surface that will not influence the dynamics of particles. The magnetic field in the particles frame is expressed as
\begin{equation}
    B'(t)=R_y(\alpha) B(t) 
\end{equation}
Where $R_y(\alpha) $ is the rotation matrix around y axis.  The dynamics of particles angular and translational velocity are solved numerically to determine the particle position $\ve{x}_{\p}(t)$ and its orientation parameterized by the unit quaternion $\ve{q}(t)=[q_0,q_1,q_2,q_3]^T$.\cite{diebel2006representing}(See the details of numerical simulation in \textcolor{red}{Appendix}). The simulated trajectory  of one applied external  magnetic field  \textcolor{red}{figure}is shown in \textcolor{red}{figure}
\begin{figure}[p]
\centering
\includegraphics[width=9cm]{figures/5_2.pdf}
\caption{ (a) Left:The force balance on the magnetic roller on an inclined surface; right: transform from world frame to particle frame. particles have 3 degree of freedom for rotation. (b)The design process for the magnetic field. The physical model can forward simulate the dynamics of rollers on the inclined surface. Moreover, the simulation results can inversely guide the design of magnetic field.}
\label{fig:1}
\end{figure}
 \section{Design the Magnetic Field}
 The desired magnetic field should be periodic that lead no net motions from cycle to cycle for particles on a horizontal surface and navigate particles gravitaxis motions on an inclined surface.
 We consider periodic magnetic fields $\ve{B}(t)$ with a fundamental frequency $\omega$ and $N$ harmonics
\begin{equation}
    \ve{B}(t) = \tfrac{1}{2} \ve{a}_0 + \sum_{n=1}^N \ve{a}_n \cos(n\omega t) + \ve{b}_n \sin(n\omega t)
\end{equation}
where $\ve{a}_n$ and $\ve{b}_n$ are constant vectors. This $6N+3$ dimensional design space is constrained by additional requirement that the field have $m$-fold rotational symmetry about the $z$-axis
\begin{equation}
    R_z(\varphi_m) \ve{B}(\omega t) = \ve{B}(\omega t - \varphi_m) \label{eq:symmetry}
\end{equation}
where $\varphi_m=2\pi/m$ for positive integer $m\geq 3$, and $R_z(\varphi)$ is the rotation matrix about the $z$-axis.
For $n=0$, equation (\ref{eq:symmetry}) implies that $\ve{a}_0 = a_0 \ve{e}_z$. For $n\geq1$, equation (\ref{eq:symmetry}) has terms like $\cos(n \omega t)$ and $\sin(n\omega t)$, which can be collected to give six linear equations for the six coefficients 
\begin{equation}
    \begin{bmatrix} 
    R_z(\varphi_m) - I \cos(n\varphi_m) & I \sin(n \varphi_m) \\
    -I \sin(n \varphi_m) & R_z(\varphi_m) - I \cos(n\varphi_m)
    \end{bmatrix} 
    \begin{bmatrix} \ve{a}_n \\ \ve{b}_n \end{bmatrix} = 0
\end{equation}
where $I$ is the identity matrix.  Importantly, these equations six equations are not always independent of each other, in which case there exists a null space of non-trivial solutions for the coefficients $\ve{a}_n$ and $\ve{b}_n$.  For example, when the field has $m$ fold rotation symmetry, the $n=1$ harmonic has coefficients of the form 
\begin{equation}
    \begin{bmatrix} \ve{a}_1 \\ \ve{b}_1 \end{bmatrix} = c_1 \begin{bmatrix} 1\\0\\0\\0\\1\\0 \end{bmatrix} + d_1 \begin{bmatrix} 0\\1\\0\\-1\\0\\0 \end{bmatrix}
\end{equation}
where $c_1$ and $d_1$ are arbitary coefficients. For a given order of rotational symmetry $m$, non-trivial solutions exists for $n=1$, $n=k m-1$, $n=k m$, and $n=k m + 1$  where $k= 1,2,\dots$ is a positive integer. For $m$-fold rotational symmetry, the corresponding null space is given by
\begin{align}
    \quad \begin{bmatrix} \ve{a}_{n} \\ \ve{b}_{n} \end{bmatrix} &= c_{n} \begin{bmatrix} 0\\0\\1\\0\\0\\0 \end{bmatrix} + d_{n} \begin{bmatrix} 0\\0\\0\\0\\0\\1 \end{bmatrix} \quad \text{for} \quad n=km
    \\
    \quad \begin{bmatrix} \ve{a}_{n} \\ \ve{b}_{n} \end{bmatrix} &= c_{n} \begin{bmatrix} 1\\0\\0\\0\\\pm1\\0 \end{bmatrix} + d_{n} \begin{bmatrix} 0\\1\\0\\\mp1\\0\\0 \end{bmatrix} \quad \text{for} \quad n=km \pm 1
\end{align}
for positive integers $k$ and $m\geq3$. To limit the size of our design space, we fix the number of higher harmonics to $N=m+1$. In this way, the full $6N+3$ dimensional design space is reduced to only 9 dimensions (namely, $a_0$, $c_1$, $d_1$ $c_{m-1}$, $d_{m-1}$ $c_{m}$, $d_{m}$, $c_{m+1}$, and $d_{m+1}$). As the phase of the driving field is not important, we set $d_1=0$ without loss of generality.  Moreover, as the field has rotational symmetry about the $z$ axis, rotation of the field about that axis by any angle is not important; we can therefore set $d_{m-1}=0$ without loss of generality.  Figure \textcolor{red}{5.1-c} illustrates one possible magnetic field with $m=10$ fold symmetry.

To design a magnetic field, we use our model to numerically simulate our particles' trajectories on the slope. The physical model works as a black box model(shown as figure\textcolor{red}{5.2}) with input as variables, $a_0$, $c_1$, $c_{m-1}$, $c_{m}$, $d_{m}$, $c_{m+1}$, and $d_{m+1}$(to simplify the notation, we re-denot the input as $\vec{v}=$ $v_1$,$v_2$,$v_3$,$v_4$,$v_5$,$v_6$,$v_7$  ). The output of the black-box function is the dynamics of particles. In order to achieve the desired gravitaxis motion, we defined the target function as,
\begin{equation}
    f(\vec{v})=|\frac{\delta x}{\delta y}| sign(\delat y)
\end{equation}
where $\delta x$ and $\delta y$ are the net displacement of particles after one period of time various magnetic field $B(t)$ parametered with $\vec{v}$.
The black-box optimization algorithm the  is applied to minimize the target function.\cite{dou2019autonomous}. Minimization of target function $f(\vec{v})$ is trying to maximum the gavitaxis motion uphill while reduce the motion perpendicular to the gravity as much as possible. The design process is shown in the \textcolor{figure }{figure } and the optimized field with desired trajectory is plotted in  \textcolor{figure }{figure}.


 
\section{Discussions}
In addition to optimize the magnetic field, we also explore how the navigation dynamics depend on different physical parameters in the system. We select 20 optimized fields and numerically study the influence of distance from particle to surface$\delta$, number of  rotational symmetry number $N$, inclined surface angle $\alpha$ and  minimum frequency of system  $\omega$. The results are plotted in \textcolor{red}{5.3}. As shown in the figure \textcolor{red}{5.3}, the navigation displacement$\delta y/a$ decreases while $\delta$ increase. The physical explanation for this is that coupling between rotation and translation motion gets weaker for larger distance between particle and surface. We also found that the navigation displacement is increasing as we increasing the   number of  rotational symmetry number $N$, inclined surface angle $\alpha$ and  minimum frequency of system  $\omega$, which can be used to increase the navigation velocity. The physical explanation between the navigation behaviors and these parameters need further theoretical study. 



\begin{figure}[p]
\centering
\includegraphics[width=9cm]{figures/5_5.pdf}
\caption{(a)Distribution of the navigation directions in the statistical study; Navigation behaviors depend on the physical properties(b) the distance between spheres and surface $\sigma$, (c) the number of summery $N$(d) the angle of slope surface $\alpha$, (e)minimum frequency of system  $\omega$ }
\label{fig:1}
\end{figure}


We also did a statistical study on how the design variables $\vec{v}$ influence the  navigation behaviors for particles. 100k random data points were generated first in seven dimension for $\vec{v}$ with the Latin-hypercube method.\cite{park1994optimal} . The the target function is evaluated 100k times via paralleling computing.\textcolor{red}{figure5} plot the pair plot(individual's kde(kernel density estimate) distribution plot and the correlation contour plot between different parameters.) of statistical study result.The right upper part plot the date corresponding to the navigation to uphills($\delta y>0$) while the left bottom part plot the date corresponding to the navigation to down hills($\delta y<0$). It is clear that the data pattern for uphill and downhill motions are different. From the the KDE plot, the parameters leading to uphill motions are distribute around mean value$\vec{v}=[0.5,0.5,0.5,0,0,0.5,0.5]$  , which try to squeeze the magnetic field shape in z-direction making the magnetic field  very flat.This is founding from statistics agree with our optimization results. As comparison, the parameters leading to downhill motions are distribute around mean value$\vec{v}=[0,0,0,1,1,0,0]$,which try to 
stretch the magnetic field shape in z-direction.
 
 
 \begin{figure}[p]
\centering
\includegraphics[width=16cm]{figures/5_4.pdf}
\caption{ Statistical study on the designed magnetic field and navigation behaviors with 100k simulations. The pair plot shows the distribution of parameters$\vec{v}$ and correlations between each design parameters for negative/positive gravitaxis separately.}
\label{fig:1}
\end{figure}
 
 
 
 \section{Preliminary experiment results and challenge}
 The same setup is adapted from our group's previous paper as shown in \textcolor{red}{fig xxx}. \cite{fei2018magneto, fei2019magneto} A 3 dimensional time variance magnetic field is generated with a three axis electric magnetic coil mounted on the microscope. The orientation of magnetic field is controlled by the wave functioned in $x$, $y$ and $z$ direction. The magnitude of magnetic field is controlled by the current. The field can be rotated by multiply a 3-d rotational matrix. 3D printed  surfaces with slope angle from $5^o$ to  $20^o$ are put under microscope to generate gravity field. We use one optimized magnetic field(10 symmetry number with the flat shape  squeezing in z direction ) to apply torque on the experimental ferro magnetic particles. The results are shown in the figure \textcolor{red}{5.5}. When the magnetic field is flat without rotation along y axis, the particles are just following the periodic magnetic field showing no directed motion. Interestingly, as we rotate the magnetic field around $20$ degree, and apply the same kind of periodic field, the particles began to show a very clear directed motion from left to right as shown in the \textcolor{red}{figure}. However, this directed motion is different from the simulation result. The un-matching results between  experiment and simulation happens for lots of experiment we tried, which seems a very big challenge. 
 
 The reason for the incorrect prediction results in our model for real experiment may be caused by the inadequate physics. There are lots of assumption in our model to simplify the questions, which may not be accurate. For example, in our model we assume the distance $\delta$ between the particle and surface is constant. Considering the constantly changing rotation direction of particles and the surface force from the slope, this constant surface separation assumption may not be accurate. \textcolor{red}{figure} has shown how $\delta$ can affect the dynamics of rollers. So if our model want to capture the real experiment behaviors, a more precise and detailed physics model should be proposed to  simulate the dynamics of micro rollers. Another reason may be from the noise of the system.There are lots of noise in the experiment system  such as the real magnitude and direction of magnetic field compared to our perfect simulation results. Our optimized algorithm only finds a good  "point" on a very noise solution space instead of a region. If some parameters are really sensitive(for example ,changing 1$\%$ amount of this parameter will lead to totally different result), it is not surprising that noise in the experiment can lead to different results from the simulation. We can make our experiment setting better by using a more accurate equipment to reduce the noise. Alternately, we can re-program the optimization algorithm to let it find the best optimized results in a region with some noise instead of a single point. However, if the optimized  surface or blob is much harder to achieve than the optimized point, we can use other shapes in other navigation system. Non-spherical shapes have very different hydrodynamic resistance tensor\cite{brooks2018shape} so that particular shape may easier to have the navigation behaviors. Another very promising approach is to use real experiment results as the output of black function and optimize the parameters dynamically with the real experiment output. This will need the automation of experiment process that can automate apply field, capture and analyse videos to generate large amount of real experiment data.\cite{oulmas20183d}

 \begin{figure}[p]
\centering
\includegraphics[width=9cm]{figures/5_6.pdf}
\caption{ Experiment showing navigation into different direction (a) Particles with no navigated motion on a planner surface doing periodic closed trajectory motion  (b)Particles with navigated motion on a inclined surface  surface. }
\label{fig:1}
\end{figure}
