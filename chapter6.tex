\chapter{Future directions and Outlook}
This thesis started with introducing the concept and road map of colloid robot research. We gave two examples in chapter 2 and chapter 3 of how to make actuators for colloid robots with a high efficient electrostatics actuation mechanism called contact charge electrophoresis(CCEP). Chapter 2 showed individual asymmetric colloid particle can be used as actuator in a very actuation rate. Chapter 3 showed complex travelling actuation for colloid robots can be achieved by coupling individual colloid particle motions together.  In addition to actuation of colloid robots, we further proposed two strategies to let colloids achieve autonomous navigation in chapter 4 and chapter 5 via sensing or interacting with the complex environment. In chapter 4, colloid robots are designed to change the shape versus the local information, which add sensing and feedback system for colloid robots. We program the shape and shape-shifting process versus different information to let colloid robot achieve global navigation with only local information.In chapter 5, we program the parameters in magnetic  actuation source of colloid robots. Colloid robots are also demonstrated to show navigation behaviors(uphill or downhill). However, the results and achievement in this thesis is just the end of beginning(or even not reach) of the colloid robot research, lots of more interesting and challenging problems need to solve in the colloid robots field. The last chapter of this thesis is going to give an outlook for the future research in colloid robots.
\section{Actuation mechanism for  colloids robots}
The future development for  actuation mechanism should focus on improving efficiency and speed as well as understanding different motion mode. Lots of current actuation mechanism has either very energy conversion low frequency or low speed motions, which is not compatible to the real microorganism such as bacteria. Also new actuation mechanism should let colloid robots have more degree of freedom to move. Colloid robots introduced in this thesis and lots of other papers have very limited degree of freedom for their motions. These limited motions are usually rolling,translating and folding which is far behind a robotics arm or hand with many joints. 

One interesting electrostatics actuation mechanism called Quincke rotation which can generate fast autonomous rotation speed with very low energy input is a promising new actuation mechanism. \cite{das2013electrohydrodynamic} Different shapes such as sphere, ellipsoid and helix have shown high motion speed in microscale.\cite{brosseau2019relating,das2019active} The complex non-linear multipole electrodynamics and hydrodynamics problems in Quincke rotation is still unsolved, needing both experimental and theoretical study. More accurate  and  less time consuming algorithm should be developed . \cite{fiore2019fast}  A better understanding of dynamics in Quincke rotation can help us design faster and more efficient colloid robots. Another main characteristics of a good actuation mechanism  should be flexible and robust. The actuation mechanism should work for a large range of materials in different fluid environments reliably.With respect to these characteristics,  magnetic actuation mechanism stand out from other mechanisms. Because the magnetic force can be well controlled without influenced by the fluid environments or materials(as the example shown in the chapter 5).

Another research direction should focus on developing well packed autonomous platform to drive the motion of colloid  robots. Most of current colloid robots are disposable products or single used, which is not sustainable. The autonomous platform should integrate the process of colloid robots. In a autonomous process, the platform can produce colloid robots and insert colloid robots into experimental environment autonomously. Then the actuation and imaging will be triggered to  use a feedback loop to actuate the microrobots. After the experiment/finishing the task, the colloid robots should be collected by the platform for future implement.The similar autonomous platform has been built to study chemical synthesis and biology research. \cite{grizou2020curious,chao2019systems} The autonomous workflow platform can not only saving time during research but also accelerates the industrialize process of colloid robots.   

\section{Swarm colloid robots}
Swarm colloid robots take advantage of the interactions between colloid robots to generate high level functions such as pattern formation and collective motion behaviors. In chapter 2 we showed linear colloid robots actuators can form travelling waves. An intuitive research direction is to explore 2 dimension electrostatic actuator's dynamics. In the experiment, we can use two parallel plate electrode and place actuators between the plate in a honeycomb or square lattice  holes to study the dynamics.(as shown in \textcolor{red}{fig}) The dynamics structures in 2D will be much more visually impressive than a 1-d travelling waves and will generate motions similar to biofilms.There are may some experimental technical challenge to put particles  in thorough micro honeycomb lattice holes. One possible solution is to use laser tweezers to place the micro particles into each hole one by one. Alternatively, it is likely to use MEMS to top-down fabricate the micro spheres in  holes structure with multiplayer's lithography, deposition and etching. To model the system, it is very important to find a suitable coarse-grain model instead of direct detailed numerical simulations of all the particles. The coarse-grain model should be simple enough that don't require too much computational source, while it should be accurate enough to capture the key physics and  behaviors in the system. The analogy is statistical mechanics that can explain the collective dynamics without considering individual agents too much.

As we discussed in the first chapter, most of the swarm colloid robots now are simply using passive interaction showing only one type of dynamics, which is not re-programmable. If the interactions between the colloid robots can be re-programmed, different mode of collective behaviors can be generated. Recent research has shown that the magnetic interactions between magnetic particles   can be programmed by introducing multi-pole interactions. They can  different  equilibrium self-assembly structure subject to different interactions. \cite{niu2019magnetic} The same programmed method can be used in non-equilibrium swarm colloid robots to generate different dynamics patterns. Moreover, collective behaviors among large amount of micro scale colloid robots  will generate impact in macro scale showing reconfigurable, adaptable and scalable characteristic,  which is similar to  how the small living cells built the whole body. Thus, swarm colloid robots can provide revolutionary new actuation mechanism for the real size robotics. 
%%%%%%%%%
\section{Shape shifting colloids robots}
To experimentally realize the navigation strategy with shape shifting colloid robots, the key is the development of shape shifting materials responding to the environmental stimulus. There several candidate materials such as  light response liquid crystal\cite{palagi2016structured}, thermal response shape memory alloy\cite{busch1991shape}: liquid crystal can change the molecule's structure upon receiving photons from  light, leading to the shape shifting of liquid crystal; the atom lattice of shape memory alloy will change due to the thermal energy and will lease back to normal in the room temperature. These materials have been reported working very well in microstructure form different structure under corresponding stimulus\cite{breger2015self}, although most papers use shape-shifting as mechanism to actuate instead of sensing.\cite{tu2017self,li2018light} A possible experiment system could use a dielectric or magnetic nanoparticle embedded light response liquid crystal materials with some designed shape.   The material's motion can be actuated via the outside electrostatic or magnetic field. With different environmental stimulus, the particles will change to different shapes simultaneously, leading to navigation behaviors shown in chapter 4. 

There several challenge to solve for the experimental system mentioned above. First challenge  is to design the shape shifting process of the colloid robots. We gave a simple example in chapter 4 on how to design this process with a reverse design optimization. However, it will be hard to built a model perfectly predict the motions of all different shape colloid in external field. Numerical models need to work together with large experiments efforts to help us fully understand how shape can direct different motions for colloid robots . \cite{lee2019directed} Then with a "dictionary" of shape and its corresponding motion, we can try to design the shape shifting process for colloid robots based on the guideline in chapter 4. Second challenge is to match the time scale between shape shifting rate and actuation speed. The changing rate of current shape shifting materials is relative very low that will take seconds or even minutes to change one shape to another. When the colloid robots sense a new environmental stimulus it can change the shape simultaneously without too  much time delay. late reaction time(or called as long memory) of shape shifting will  lead to wrong navigation direction of colloid robots.Future research should focus on developing materials with fast changing rate. The third challenge is to built a microscale stimulus landscape. The gradient of stimulus landscape should be small enough compared to the size of colloid robots that colloid robots only feel an uniform stimulus in one location. This need a very precise  design of the mirco scale topological structure such as micro patterned light grid to control the light intensity. Further more, shifting materials are working as sensors which means if other internal states can change with the stimulus from  environments, these states can also be used as sensors for colloid robots. For example if the charge amount, some chemical species, strength of magnetic dipole on the colloid robots can change versus the environmental stimulus, all of these can be used as sensors to design the autonomous navigation strategy for colloid robots.  




\section{Programmable and design the colloids robots}
For the purpose of engineering application and larger impact, colloid robots must be programmed and designed rationally, which is still a relative blank research area. This is because most researchers in colloid robots are from natural science field like chemistry and physics. It is a pity that colloid robots haven't arise too much attention in robotics research community from electrical engineering or computer science researchers, who focus more on the programming and design problems.\cite{das2019cellular}  Lots of control strategy, design approach and program algorithm in large scale robots can help a lot in the research on colloid robots.  The future research on colloid robots should involve intensive collaborations between robotics field and colloid fields.

One very promising  direction is to machine learning or artificial intelligent guiding the design of colloid robots. There are lots of design parameters for a colloid robots such as materials choice, shape design, magnitude and form of power source. Experimentalist will have tons of data  either successful data or failure data during the research. Machine learning approach can make these data more valuable. With clustering, classification and regression for the experimental data, machine learning(or deep learning) algorithm can not only find the pattern to design a good colloid robots, but also be likely to find some nontrivial design rules to colloid robots. In addition to use study amount of data with machine learning, optimization algorithm can help us design colloid robots with less time consuming experiment approach. If we know the design space for the colloid robots, either with the experimental or simulation output, optimization algorithm(such as CME-ES algorithm in chapter 4) can guide the selection of next design parameters in the design space. Recent reports also showed Bayesian inference can help to estimate design parameters with less experiments. \cite{winslow2019characterization}



There are more functions beyond autonomous motion and navigation could be designed for colloid robots. Cargo transportation is now an emerging research direction.\cite{demirors2018active,Martinez-Pedrero2015} Basically, the colloid robots can use physical interactions from electrostatic, magnetic to hydrodynamics to attract and capture cargo. The colloid robots will move with cargos along some designed route and release the cargo in the destination via turning the interaction off. The future research could focus  on the developing method for cargo sorting and cargo assembling like an Amazon robot in microscale.


\section{Application of colloid robots}
At the current stage, all of the colloid robots' applications are still being explored in the lab. But with the unique properties of colloid robots including cell scale size and different dynamics function, the potential applications for colloid robots are very impressive. In the last section of this thesis, I will briefly outlook the applications for colloid robots.

\textbf{Biomedical applications} The biggest and most intuitive applications for the colloid robots are medical applications. Colloid robots are usually in the size of living cells so that they may work as "micro" surgery knifes to remove tumor, transport drug or repair wound for patience. Several in vivo colloid robot experiments  on animals have been reported \cite{Gao2015,li2018development} with promising results. There are several very strict standard must be satisfied when we design a colloid robots for biomedical applications. First the materials of colloid robots must be bio compatible that have no harm to the living system and have a pathway to be removed from the living body or even can be digested by the living body without any side effects.
The materials should also be modified to have proper chemical or physical affinity so that colloid robots can carry drugs or attack the tumors in the living system. Some candidate materials could be silica gel materials(PDMS) or some nontoxic metals.This need tremendous in vivo experiments on animals before  any possible experiments on human being. Second the actuation mechanism for the colloid robots must also be safe enough. For example, if we use magnetic or electric field, the intensity of external field shouldn't exceed the maximum bearing value of humans. These fields also shouldn't be  filtered or screened by some structures inside human body. So may magnetic field or acoustic field that have already by used in medical treatment are good  candidate actuation mechanism.  Third , the cost of the whole colloid robot's system should be kept as affordable as possible so that colloid robots can have a widely applications for most patients. Research on colloid robots for biomedical applications should involve collaborations from doctors and scientist in medical school to get their insights and knowledge.

\textbf{Other applications} Colloid robots have been demonstrated with other interesting applications such as cleaning water, mixing fluid and lithography. \cite{soler2014catalytic,fei2019magneto,li2014nanomotor}. These applications are all depend on the autonomous motion and navigation of colloid robots. As a chemical engineer, I would like to propose an interesting applications in chemical engineering field: colloid robots may be a new technology to enhanced oil recovery from ground. In petroleum industry, large amount of crude oil(more than 50$\%$) are not extracted from rock because they are in some dead end micro pores with very low Reynolds number. Colloid robots with autonomous motion and navigation behaviors can be designed with good crude oil affinity via chemical/physical modification. Then we can send these colloid robots into the micro pore full of oil.  Then colloid robots  can collect the oil in micro poles and be navigated back with these crude oil. In a short summary, colloid robots seems to be something from science fiction, but it is really happening now with many scientists and engineers from different fields working on it. Although we are just at the very early stage of colloid robots, colloid robots will have a predicable huge development with more efforts on both experiment and theory research
