\chapter{Future directions and Outlook}
This thesis started with introducing the concept and road map of colloid robot research. We gave two examples in chapter 2 and chapter 3 of how to make actuators for colloid robots with a high efficient electrostatics actuation mechanism called contact charge electrophoresis(CCEP). Chapter 2 showed individual asymmetric colloid particle can be used as actuator in a very actuation rate. Chapter 3 showed complex travelling actuation for colloid robots can be achieved by coupling individual colloid particle motions together.  In addition to actuation of colloid robots, we further proposed two strategies to let colloids achieve autonomous navigation in chapter 4 and chapter 5 via sensing or interacting with the complex environment. In chapter 4, colloid robots are designed to change the shape versus the local information, which add sensing and feedback system for colloid robots. We program the shape and shape-shifting process versus different information to let colloid robot achieve global navigation with only local information.In chapter 5, we program the parameters in magnetic  actuation source of colloid robots. Colloid robots are also demonstrated to show navigation behaviors(uphill or downhill). However, the results and achievement in this thesis is just the end of beginning(or even not reach) of the colloid robot research, lots of more interesting and challenging problems need to solve in the colloid robots field. The last chapter of this thesis is going to give an outlook for the future research in colloid robots.
\section{Actuation for  colloids robots}
\section{SWARM ROBOTS}
explore the actutaion mode as much as possible 
\section{Shape shifting colloids robots}
chapter 1
DIFFERENT SHAPE
Some research have shown some active matter can do self-regulate motion by adapting its shape with the environment conduction such as temperature and light intensity.\cite{tu2017self,li2018light}.


\section{programmble functions for colloids robots}
chapter 2
AI AND MACHINE LEARNING,
BRING THE ROBOTS RESEARCH FIELD
\section{Application of colloid robots}
Biomedical

