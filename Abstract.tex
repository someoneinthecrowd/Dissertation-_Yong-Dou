%Abstract Page
% give abstract to every single chapter
%%space ()
%% e.g., 
%% space nm 
%%Something else to note about the name "colloidal robot".  To me, it denotes (1) the size (nm to micron), (2) the environment (fluids or squishy stuff, not the vacuum environment of MEMS), and (3) the toolbox of colloid science need needed to understand and engineer functions on these scales in these environments (e.g., hydrodynamics, Brownian motion, phoretic flows, electrokinetics, etc.)


\begin{titlepage}
\begin{center}

\textbf{\large Abstract}

\textbf{Colloidal Robotics: autonomous propulsion and navigation of active particles}

Yong Dou
\end{center}

\hspace{5mm}Colloidal robots refer to the colloid scale (from nm to $\mu$m) machines capable of carrying out programmed actions for complex tasks automatically.   Colloidal robots are designed to mimic the dynamic behaviors of living cells such as autonomous motion, pattern formation, and navigation. Because of colloidal robot's promising application in engineering and medical service, namely drug delivery and single cell surgical, colloidal robots have been of much recent research interest in the context of both theoretical and technological relevance. Although many mechanisms and new materials have been developed to build and control colloidal robots over the last decade, there remain many open challenges on increasing actuation efficiency, achieving high level tasks (e.g., autonomous navigation), etc. This dissertation, in general, focuses on developing new actuation mechanisms and designing  autonomous navigation strategies for colloidal robots with both experimental and computational efforts.

Firstly, the motivation, background and recent research  advances on colloidal robots are reviewed. Particularly, we emphasize the importance of feedback
loop composed of actuators, sensors, and controllers for colloidal robots.  In \textbf{Chapter 2},  a high-efficiency actuation method called contact charge electrophoresis(CCEP) is introduced to propel the dielectric metallic Janus colloid particles.  The autonomous propulsion of Janus colloidal particles shows colloidal particle asymmetries can be used to direct the motions of colloidal robots. Beyond single colloidal particle's propulsion, \textbf{Chapter 3} shows multi-colloidal particles' motions can be coupled and synchronized to generate  traveling waves via electrostatic interactions.   Our results in Chapter 3 suggest that simple energy inputs can coordinate complex motions with opportunities for colloid machines.  Then inspired by active particles motions' guided by their symmetry in Chapter 2, we show in \textbf{Chapter 4} how multiple autonomous navigation can be achieved by designing the active particle's geometry and its stimulus response. Chapter 4 describes a strategy that colloid particles can sense the stimulus in environment via shape-shifting. The feedback loop of sensing and motion enables colloid particles to achieve positive or negative chemotaxis-like navigation for colloidal robots. To experimentally realize similar navigation behaviors introduced in Chapter 4,  we described a magnetic driven colloidal robot system in \textbf{Chapter 5}, which could show navigation behaviors (uphill and downhill) on a slope by rationally  programming the external magnetic field. \textbf{Chapter 6} highlights future research  directions motivated by this dissertation and potential applications of colloidal robots.

%While Chapter 2 and Chapter 3 are only focusing on motions of active particles,
\vspace*{\fill}
\end{titlepage}
