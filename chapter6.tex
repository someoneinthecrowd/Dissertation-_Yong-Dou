\chapter{Colloidal robots: the end of the beginning }
\begin{center}
\textbf{Abstract}
\end{center}
This thesis started with introducing the motivation and background of colloidal robot research. A fully autonomous colloidal robot should have feedback composed of actuators, sensors, and controllers.  In chapter 2 and chapter 3, we gave two examples on how to make actuators for colloidal robots with a high efficient electrostatics actuation mechanism called contact charge electrophoresis (CCEP). Chapter 2 showed the asymmetry of the colloid particle can provide a large design space for different kinds of actuation motions in colloidal robots.  Chapter 3 showed complex traveling actuation for colloidal robots can be achieved by coupling individual actuator's motion together.  In addition to actuation, we further proposed two strategies to design sensors and controllers for colloidal robots. With actively interacting the with environment through the feedback system, colloidal robots can finish some complex targets such as autonomous navigation. In chapter 4, colloidal robots are designed to alter the shape versus the local information or stimulus. We programmed the shape and shape-shifting process versus different stimuli to let colloidal robots achieve global navigation with only local information. In chapter 5, we programmed the parameters of the external magnetic field used to actuated colloidal robots. colloidal robots are also demonstrated to show navigation behaviors  (uphill or downhill) on an inclined. However, the results and achievement in this thesis is just the end of the beginning (or even not reach) of the colloidal robot research, lots of more interesting and challenging problems need to solve in the colloidal robots field. The last chapter of this thesis is going to give an outlook for the future research direction in colloidal robots.
\section{Actuation mechanism for  colloids robots}
Future development for  the actuation mechanism should focus on improving efficiency, speed, and robustness as well as understanding the detailed physics behind different mechanisms. Lots of current actuation mechanism has either very low energy conversion ($<0.1\%$) efficiency due to a large amount of heat dissipation or very slow speed motions, which is not compatible to the real living machines such as bacteria. New actuation mechanisms should also let colloidal robots have more degrees of freedom to move. colloidal robots introduced in this thesis and lots of other papers have very limited degrees of freedom for their motions. These motions are usually simply rolling, translating or turning around which is far from the various actuation mode in a robotic arm or hand with many joints. 

One interesting electrostatics actuation mechanism called Quincke rotation which can generate fast autonomous rotation between the electrode with very low energy input is a promising new actuation mechanism \autocite{das2013electrohydrodynamic}. Different shapes such as spheres, ellipsoids and helix have shown high actuation speed in microscale\autocite{brosseau2019relating,das2019active}. A better understanding of dynamics in Quincke rotation can help us design faster and more efficient colloidal robots. However, the complex non-linear multipole electrodynamics and hydrodynamics problems in Quincke rotation are still unsolved, requiring  both experimental and theoretical study. A accurate  and  less time-consuming algorithm should be developed\autocite{fiore2019fast}. Other important characteristics of a good actuation mechanism  is flexible and robust. The actuation mechanism should work for a large range of materials in different fluid environments reliably. With respect to these characteristics,  the magnetic actuation mechanism stands out from other mechanisms. Because the magnetic force can be well controlled without influenced by the fluid environments or materials (as the example shown in chapter 5).

Another research direction is developing a well packed autonomous platform to drive the motion of colloid  robots.  The autonomous platform should integrate the process in colloidal robots research from fabrication, experiment to analysis and recycling. In an autonomous process, the platform can produce colloidal robots and insert colloidal robots into the experimental environment autonomously. Then the actuation and imaging will be triggered to  use a feedback loop to actuate the microrobots. After the experiment/finishing the task, the colloidal robots should be collected by the platform for future implementations. A similar autonomous platform has been built to study chemical synthesis and biology research \autocite{grizou2020curious,chao2019systems}. The autonomous workflow platform can not only saving time during research but also accelerates the industrialize process of colloidal robots.   

\section{Swarm colloidal robots}
Swarm colloidal robots take advantage of the interactions between colloidal robots to generate high-level functions such as pattern formation and collective motion behaviors. In chapter 3 we showed linear colloidal robots actuators can form travelling waves. An intuitive research direction is to explore 2 dimension electrostatic actuator's dynamics with the same system in chapter 3. For the experiment, we can use two parallel plate electrodes and place actuators in a honeycomb or square lattice  holes to study the dynamics. The dynamics structures in 2D will be much more visually impressive than  1-d travelling waves and could generate motions similar to biofilms. There are some experimental technical challenges to put particles  inside micro honeycomb lattice holes. One possible solution is to use laser tweezers to place the micro particles into each hole one by one. Alternatively, we can also MEMS 's top-down fabrication to make the micro spheres in  holes structure with multiplayer's lithography, deposition, and etching. To model the system, it is very important to find a suitable coarse-grain model instead of direct detailed numerical simulations of all the particles. The coarse-grain model should be simple enough that doesn't require too much computational source, while it should be accurate enough to capture the key physics and  behaviors in the system similar to the model proposed in chapter 3. The analogy is statistical mechanics that can explain the collective dynamics without considering individual agents too much.

As we discussed in the first chapter, most of the swarm colloidal robots now are simply using passive interaction showing only one type of dynamics, which is not re-programmable. If the interactions between the colloidal robots can be re-programmed, a different mode of collective behaviors can be generated. Recent research has shown that the magnetic interactions between magnetic particles   can be programmed by introducing multi-pole interactions. They can  different  equilibrium self-assembly structures subject to different interactions\autocite{niu2019magnetic}. The same programmed method can be used in non-equilibrium swarm colloidal robots to generate different dynamics patterns. Moreover, collective behaviors among large amounts of micro scale colloidal robots  will generate impact in macro scale showing reconfigurable, adaptable and scalable  characteristics,  which is similar to  how the small living cells built the whole body. Thus, swarm colloidal robots can even provide revolutionary new actuation mechanism for the real size robotics. 
%%%%%%%%%
\section{Shape shifting colloids robots}
To experimentally realize the navigation strategy with shape-shifting colloidal robots in chapter 4, the key is the development of shape-shifting materials responding to the environmental stimulus. There several candidate materials such as  light response liquid crystal\autocite{palagi2016structured}, thermal response shape memory alloy\autocite{busch1991shape}. Liquid crystal can change the molecule's structure upon receiving photons from  light, leading to the shape shifting of liquid crystal; the atom lattice of shape memory alloy will change due to the thermal energy and will lease back to normal in the room temperature. These materials have been reported working very well in microscale to form different structure under stimulus\autocite{breger2015self}, although most papers use shape-shifting as a mechanism to actuate instead of sensing\autocite{tu2017self,li2018light}. A possible experiment system could use a dielectric or magnetic nanoparticle embedded light response liquid crystal materials with some designed shape.   The material's motion can be actuated via the outside electrostatic or magnetic field. With different environmental stimuli, the particles will change to different shapes simultaneously, leading to navigation behaviors shown in chapter 4. 

There several challenges to be solved for the experimental system mentioned above. The first challenge  is to design the shape shifting process of colloidal robots. We gave a simple example in chapter 4 on how to design this process with the reverse design optimization. However, it will be hard to build a model perfectly predicting the motions of all different shape colloids in external field. Numerical models need to work together with large experiments efforts to help us fully understand how shape can direct different motions for colloidal robots\autocite{lee2019directed}. Then with a "dictionary" of shape and its corresponding motion, we can try to design the shape shifting process for colloidal robots based on the guideline in chapter 4. The second challenge is to match the time scale between shape shifting rate and actuation speed. The changing rate of current shape shifting materials is relatively very low that will take seconds or even minutes to change one shape to another. When the colloidal robots sense a new environmental stimulus it should change the shape simultaneously without too  much time delay. Late reaction time (or called as long memory) of shape shifting will  lead to the wrong navigation direction of colloidal robots. Future research should focus on developing materials with fast-changing rate. The third challenge is to built a microscale stimulus landscape. The gradient of stimulus landscape should be small enough compared to the size of colloidal robots that colloidal robots only feel a uniform stimulus in one location. This requires a very precise  design of the mirco scale topological structure such as micro patterned light grid to control the light intensity. Shifting materials are working as only sensors. So to extend this idea, if other internal states can change with the stimulus from the environments, these states can also be used as sensors for colloidal robots. For example, if the charge amount, some chemical species, the strength of magnetic dipole on the colloidal robots can change versus the environmental stimulus, all of these can be used as sensors to design the autonomous navigation strategy for colloidal robots.  




\section{Programmable and design the colloids robots}
For engineering applications and commercialized requirements, colloidal robots must be programmed and designed rationally, which is still a relatively blank research area. This is because most researchers in colloidal robots are from natural science fields like chemistry and physics. It is a pity that colloidal robots haven't arisen too much attention in robotics research community such as electrical engineer or computer scientist, who focus more on the programming and design problems\autocite{das2019cellular}.  Lots of control strategy, design approach and program algorithm in large scale robots can help a lot in the research on colloidal robots.  Future research on colloidal robots should involve intensive collaborations between robotics field and colloid fields.

One very promising  direction is to use machine learning or artificial intelligent guiding the design of colloidal robots. There are lots of design parameters for a colloidal robot such as materials choice, shape design, magnitude and form of the power source. The typical way for experimentalists is to do tons of repeating experiment either successful approaches or failed attempts during the research. The machine learning approach can make failure data more valuable. With clustering, classification and regression for the experimental data, machine learning (or deep learning) algorithm can not only find the pattern to design a good colloidal robot, but also be likely to find some nontrivial design rules to colloidal robots. In addition to studying large amounts of data with machine learning, optimization algorithm (or optimal design) can help us design colloidal robots with less time-consuming experiment approaches. If we know the design space for the colloidal robots, either with the experimental or simulation output, optimization algorithms (such as CME-ES algorithm in chapter 4) can guide the selection of next design parameters in the design space. Recent reports also showed Bayesian inference can help to estimate design parameters with fewer experiments \autocite{winslow2019characterization}.



Also, there are more functions beyond autonomous motion and navigation could be designed for colloidal robots. Cargo transportation is now an emerging research direction\autocite{demirors2018active,Martinez-Pedrero2015}. Basically, the colloidal robots can use physical interactions from electrostatic, magnetic to hydrodynamics to attract and capture cargo. The colloidal robots will move with cargos along some designed route and release the cargo in the destination via turning the interaction off. Future research could focus  on the developing method for cargo sorting and cargo assembling like an Amazon robot in microscale.


\section{Application of colloidal robots}
At the current stage, all of the colloidal robots' applications are still being explored in the lab. But with the unique properties of colloidal robots including cell scale size and different dynamics function, the potential applications for colloidal robots are very impressive. In the last section of this thesis, I will briefly outlook the applications for colloidal robots.

\textbf{Biomedical applications} The biggest and most intuitive applications for the colloidal robots are medical applications. colloidal robots are usually in the size of living cells so that they may work as "micro" surgery knifes to remove tumors, transport drug or repair wound for patience. Several in vivo colloidal robot experiments  on animals have been reported \autocite{Gao2015,li2018development} with promising results. There are several very strict standard must be satisfied when we design a colloidal robot for biomedical applications. First, the materials of colloidal robots must be biocompatible  having no harm to the living system and have a pathway to be removed from the living body or even can be digested by the living body without any side effects.
The materials should also be modified to have proper chemical or physical affinity so that colloidal robots can carry drugs or attack the tumors in the living system. Some candidate materials could be silica gel materials (PDMS) or some nontoxic metals.This requires tremendous in vivo experiments on animals before  any possible experiments on the human being. Second, the actuation mechanism for the colloidal robots must also be safe enough. For example, if we use a magnetic or electric field, the intensity of the external field shouldn't exceed the maximum bearing value of humans. These fields also shouldn't be  filtered or screened by some structures inside the human body. So may magnetic field or acoustic field that have already by used in medical treatment are good  candidate actuation mechanism.  Third, the cost of the whole colloidal robot's system should be kept as affordable as possible so that colloidal robots can have wider applications for most patients. Research on colloidal robots for biomedical applications should involve collaborations from doctors and scientists in medical school to get their insights and knowledge.

\textbf{Other applications} colloidal robots have been demonstrated with other interesting applications such as cleaning water, mixing fluid and lithography\autocite{soler2014catalytic,fei2019magneto,li2014nanomotor}. These applications all depend on the autonomous motion and navigation of colloidal robots. As a chemical engineer, I would like to propose an interesting application in the chemical engineering field: colloidal robots may be a new technology to enhanced oil recovery from the ground. In the petroleum industry, a large amount of crude oil (more than 50$\%$) is not extracted from rocks because they are in some dead-end micro pores with very low Reynolds number. colloidal robots with autonomous motion and navigation behaviors can be designed with good crude oil affinity via chemical/physical modification. Then we can send these colloidal robots into the micro pore full of oil.  Then colloidal robots  can collect the oil in micro poles and be navigated back with the crude oil.

As a short summary, colloidal robots seem to be something from science fiction, but it is really happening now with many scientists and engineers from different fields working on it. We have some basics theoretical frameworks and some simple experimental demonstrations. The author has a very optimistic expectation for the future fast development of colloidal robots. Maybe in five or ten years, colloidal robots can be commercialized and serve people in different areas.  


