%Abstract Page

\begin{titlepage}
\begin{center}

\vspace*{1\baselineskip}
\textbf{\huge Abstract}

\textbf{Towards fully autonomous  colloids robots: from motion to navigation}

Yong Dou
\end{center}

\hspace{5mm}Colloid robots refer to the colloid scale(from $nm$ to $\mu m$) machines capable of carrying out programmed actions for complex tasks automatically.   Colloid robots are designed to mimic the dynamic behaviors of living cells such as autonomous motion, pattern formation, and navigation. Because of the colloid robot's promising application in engineering and medical service, namely drug delivery and single cell surgical, colloid robots have been of much recent research interest in the context of both theoretical and technological relevance. Although many mechanisms and new materials have been developed to build and control colloid robots over the last decade, there remain many open challenges on increasing actuation efficiency, achieving high level tasks(e.g. autonomous navigation), etc. This dissertation, in general, focuses on developing new actuation mechanisms and designing  autonomous navigation strategies for colloid robots with both experimental and computational efforts.

Firstly, the motivation, background and recent research  advances on colloid robots are reviewed. Then in the second chapter, a high-efficiency  actuation method called contact charge electrophoresis(CCEP) for colloid robots is introduced to actuate dielectric metallic Janus colloid particles. We characterize the actuation motions and propose a mechanism based on the rotation-induced translation of the particle following charge transfer at the electrode surface. The propulsion mechanism is supported both by experiments with fluorescent particles that reveal their rotational motions and by simulations of CCEP dynamics that capture the relevant electrostatics and hydrodynamics. The results demonstrate how particle asymmetries can be used to direct the motions of colloid robots by CCEP. In chapter 3, we show that raveling waves of mechanical actuation for colloid robots can be realized within linear arrays of conductive particles that oscillate between biased electrodes through cycles of contact charging and electrostatic actuation. The repulsive interactions among the particles along with spatial gradients in their natural frequencies lead to phase-locked states characterized by gradients in the oscillation phase. The frequency and wavelength of these traveling waves can be specified independently by varying the applied voltage and the electrode separation. We demonstrate how traveling wave synchronization can enable the directed transport of material cargo. Our results suggest that simple energy inputs can coordinate complex motions with opportunities for colloid robots. In chapter 4, we describe a method for programming the autonomous navigation of colloid robots in response to spatial gradients in a scalar stimulus. Functional behaviors such as positive or negative chemotaxis are encoded in the particle shape, which responds to the local stimulus and directs self-propelled particle motions. We demonstrate this approach using a physical model of stimuli-responsive clusters of self-phoretic spheres. We show how multiple autonomous behaviors can be achieved by designing the particle geometry and its stimulus response. In chapter 5, we described a magnetic driven colloid robot system with navigation behaviors(uphill and downhill) on a slope by rationally  programming the external magnetic field. 
The final chapter  highlights future research  directions motivated by this dissertation and potential applications of colloid robots.

\vspace*{\fill}
\end{titlepage}
